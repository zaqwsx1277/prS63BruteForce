\subsection*{Описание примера\+:}


\begin{DoxyItemize}
\item В сети есть сервер к которому подключаются клиенты.
\item Сервер не только распределяет блоки ключей выдаваемые клиентам, но и сам выступает в качестве клиента.
\item После запуска клиенты сами выполняют поиск сервера. Для этого номер порта должен быть уникальный в ЛВС или адрес и порт можно ввести вручную.
\item Клиенты выполняют подбор ключа для первых 8ми байтам файла зашифрованного по алгоритму Blowfish и если на выходе получается число 0x04034b50, то этот ключ передается серверу для окончательной проверки и дешифрации всего файла.
\item Если проверка прошла успешно, то всем остальным клиентам передается команда на прекращение подбора.
\end{DoxyItemize}

\subsection*{В данном примере показаны\+:}


\begin{DoxyItemize}
\item Знания с++11;
\item ООП;
\item Создание кросплатформенных приложений (Debian Linux, Astra Linux SE 1.\+6, Window 10 (7);
\item Работа с многопоточностью (std\+::thread);
\item Работа приложений по принципу клиент-\/сервер (Qt\+Network, Q\+Tcp\+Server, Q\+Tcp\+Socket и т.\+д.);
\item Работа с системами шифрирования данных (библиотека Botan);
\item Создание пользовательских интерфейсов;
\item Умение работать с документацией на английском языке (Изучено описание схемы защмты данных для морских карт S63 и описание протокола передачи данных S57 www.\+iho.\+int).
\end{DoxyItemize}

Вся документация к проекту и создаваемым классам выполнена по стандарту doxygen. 