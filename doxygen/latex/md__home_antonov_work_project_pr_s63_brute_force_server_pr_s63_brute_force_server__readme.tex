\subsection*{Прием/передача данных в распределённой системе\+:}


\begin{DoxyItemize}
\item После нахождения сервера, клиент посылает запрос на подключение cmd\+Connection\+Request.
\item Сервер либо подтверждает подключение cmd\+Connection\+Confirm и в этом же кадре передает клиенту размер обрабатываемого блока, либо отказывается от подключения (условия такого отказа пока неопределены).
\item При успешном подключении сервер выделяет каждому клиенту свой поток thread\+Client в котором будет вестись обмен информацией с клиентом.
\item У сервера существует контейнер std\+::set в котором хранится вся информация о потоке, ключом у которого является id-\/потока. Т.\+к.\+для обеспечения потокобезопасности данного контейнера при работе с данными блокируется любой доступ нему, то предприняты специальные меры по минимизации времени данной блокировки.
\item Получив/передав данные thread\+Client записывает эти данные в контейнер std\+::quere и запускает менеджер обработки данных выполняемый в отдельном потоке thread\+Data\+Manager. В данный контейнер так же записываются сообщения об ошибках.
\item Если запущен хоть один thread\+Data\+Manager, то он выполняет обработку всех команд записаных в контейнер и только обработав всю очередь завершает свою работу, при этом другие thread\+Data\+Manager для обработки данных не запускаются
\item При завершении работы сервера, выполняется ожидание завершения всех потоков и обработки всех команд. 
\end{DoxyItemize}